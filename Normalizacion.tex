\documentclass{article}
\usepackage[utf8]{inputenc}

\title{NORMALIZACION}
\author{Equipo: Cruz Galván Alberto Israel y Nava Escobar José Alfredo}
\date{17/03/2020}

\begin{document}

\maketitle

\section{PRIMERA 1FN}
\\ 
Se dice que una Tabla se encuentra en Primera Forma Normal si y sólo si todos sus campos (atributoss), contienen valores atómicos. Esto quiere decir que cada atributo de la tabla deberá contener un único valor para una ocurrencia de la Entidad. No se permitirán grupos repetidos.
 
\\\\
\section{SEGUNDA 2FN}
Una tabla está en Segunda Forma Normal si y sólo si está en 1FN y todos los atributos no clave dependen por completo de la clave primaria. 

\\\\
\section{TERCERA 3FN}
Una tabla está en Tercera Forma Normal si y sólo si está en 2FN y los atributos no clave son independientes entre sí. Esto quiere decir que los valores de los atributos dependen sólo de la clave primaria y no dependen de otro Atributo no clave. El valor del Atributo no debe depender del valor de otro Atributo no clave.

\\\ \\\\ \\\\ 
Consultado el día 17 de marzo de 2020 desde la dirección URL:\\\  https://ed.team/blog/normalizacion-de-bases-de-datos

\end{document}
